% This must be in the first 5 lines to tell arXiv to use pdfLaTeX, which is strongly recommended.
\pdfoutput=1
% !TeX root = stats_paper.tex

% In particular, the hyperref package requires pdfLaTeX in order to break URLs across lines.

\documentclass[11pt]{article}

\usepackage{ACL2023}

% Standard package includes
\usepackage{times}
\usepackage{latexsym}

% For proper rendering and hyphenation of words containing Latin characters (including in bib files)
\usepackage[T1]{fontenc}
% For Vietnamese characters
% \usepackage[T5]{fontenc}
% See https://www.latex-project.org/help/documentation/encguide.pdf for other character sets

% This assumes your files are encoded as UTF8
\usepackage[utf8]{inputenc}

% This is not strictly necessary, and may be commented out.
% However, it will improve the layout of the manuscript,
% and will typically save some space.
\usepackage{microtype}

% This is also not strictly necessary, and may be commented out.
% However, it will improve the aesthetics of text in
% the typewriter font.
\usepackage{inconsolata}
\usepackage{booktabs}

% If the title and author information does not fit in the area allocated, uncomment the following
%
%\setlength\titlebox{<dim>}
%
% and set <dim> to something 5cm or larger.

\title{A Cash Time Machine: Free-Cash-Flow Growth as a Predictor of Next-Year Stock Price Change}

% Author information can be set in various styles:
% For several authors from the same institution:
\author{Maya Aharon \quad Ohad Shushan \quad Itay Ebenspander \quad Almog Tavor \\
        Tel Aviv University \\}


\begin{document}
\maketitle
\begin{abstract}
We examine whether the annual growth rate of free-cash-flow (FCF) per share helps forecast the subsequent 12-month percentage price change of U.S. listed equities. Using quarterly filings from 2020-2024, we aggregate each firm's trailing-twelve-month FCF-per-share series, compute its year-over-year growth (X), and align that figure with the forward price return (Y). Performing statistical analysis over this dataset.
\end{abstract}

\section{Introduction}

To do.

\section{Data Collection and Preparation}

We assembled a panel dataset of quarterly financials and stock prices for over 5,000 U.S. companies (2020-2024) using SimFin's free API. From each company's cash-flow statement, we extracted the cash generated by core operations and subtracted any spending on long-lived assets (e.g. buildings, machinery) - this yields Free Cash Flow (FCF). We then divided FCF by the number of shares outstanding to obtain FCF per share, and computed its year-over-year growth by comparing each quarter to the same quarter one year earlier (Year over year change - YoY).

We matched each quarterly report to the first available stock-closing price within seven calendar days, and then measured YoY price growth as the percentage change in that price relative to the price one year before. A company whose share price rose from \$50 to \$100 over a year, for example, has a YoY price growth of 100\%. Finally, we computed Market Cap (price x shares outstanding) at each report date and merged all series into a single CSV file with one row per ticker-quarter, ready for statistical analysis.

\textbf{Challenges \& Limitations.}  Handling Capital Expenditure (CapEx) requires detecting different line-item names; aligning report dates to trading days can miss if a quarter falls in a long holiday; and delisted firms disappear from the price feed, so our data is thick with observations but most dense for 2021-2024, when coverage and SimFin's data availablity were highest.  

\begin{table*}[h]
  \setlength{\tabcolsep}{4pt}
  \centering
\caption{Sampled rows from our dataset. Year over year (YoY) FCF and stock prices growth for AMZN (2021-2023) and GOOG (Q3-Q4 2021), illustrating the link between cash-flow swings and price movements; the full dataset spans thousands of U.S. stocks over multiple quarters, all in USD.}
  \label{tab:sample-data}
  \begin{tabular}{lccrrrr}
    \toprule
    Ticker & Report Date & Price & YoY Price Growth & Market Cap (bn) & FCF (bn) & YoY FCF Growth \\
    \midrule
    AMZN & 2021-12-31 & 166.72 & 2.38\%   & 1,683.87 &  8.808 &  -28.03\% \\
    AMZN & 2022-12-31 &  85.82 & -48.52\% &   876.22 & 17.905 &  101.09\% \\
    AMZN & 2023-12-31 & 149.93 & 74.70\%  & 1,538.19 & 29.112 &   61.81\% \\
    \midrule
    GOOG & 2021-09-30 & 132.47 & 81.37\%  & 1,763.86 & 18.720 &   64.74\% \\
    GOOG & 2021-12-31 & 143.82 & 65.18\%  & 1,884.32 & 18.551 &   11.24\% \\
    \bottomrule
  \end{tabular}
\end{table*}

\section{Linear Regression Analysis}

We estimated the linear relation between next-year price growth ($Y$) and year-over-year free cash flow per share growth ($X$) using OLS, trimming firms with $|X|>100\%$ to exclude extreme cases and scaling $X$ to percentage points for interpretability. After these adjustments, the slope $\hat\beta_1$ quantifies the average change in return per 1 percentage point increase in FCF growth. A hexbin density plot and fitted line visually confirm the near-zero slope in noisy data. This ensures outlier control and clear economic interpretation.

\section{Document Body}

To fill.

\subsection{Footnotes}

Footnotes are inserted with the \verb|\footnote| command.\footnote{This is a footnote.}

\subsection{Tables and figures}

See Table~\ref{tab:accents} for an example of a table and its caption.
\textbf{Do not override the default caption sizes.}

\subsection{Hyperlinks}

Users of older versions of \LaTeX{} may encounter the following error during compilation: 
\begin{quote}
\tt\verb|\pdfendlink| ended up in different nesting level than \verb|\pdfstartlink|.
\end{quote}
This happens when pdf\LaTeX{} is used and a citation splits across a page boundary. The best way to fix this is to upgrade \LaTeX{} to 2018-12-01 or later.

\subsection{Citations}



Table~\ref{citation-guide} shows the syntax supported by the style files.
We encourage you to use the natbib styles.
You can use the command \verb|\citet| (cite in text) to get ``author (year)'' citations, like this citation to a paper by \citet{Gusfield:97}.
You can use the command \verb|\citep| (cite in parentheses) to get ``(author, year)'' citations \citep{Gusfield:97}.
You can use the command \verb|\citealp| (alternative cite without parentheses) to get ``author, year'' citations, which is useful for using citations within parentheses (e.g. \citealp{Gusfield:97}).

\subsection{References}

\nocite{Ando2005,augenstein-etal-2016-stance,andrew2007scalable,rasooli-tetrault-2015,goodman-etal-2016-noise,harper-2014-learning}

The \LaTeX{} and Bib\TeX{} style files provided roughly follow the American Psychological Association format.
If your own bib file is named \texttt{custom.bib}, then placing the following before any appendices in your \LaTeX{} file will generate the references section for you:
\begin{quote}
\begin{verbatim}
\bibliographystyle{acl_natbib}
\bibliography{custom}
\end{verbatim}
\end{quote}
You can obtain the complete ACL Anthology as a Bib\TeX{} file from \url{https://aclweb.org/anthology/anthology.bib.gz}.
To include both the Anthology and your own .bib file, use the following instead of the above.
\begin{quote}
\begin{verbatim}
\bibliographystyle{acl_natbib}
\bibliography{anthology,custom}
\end{verbatim}
\end{quote}
Please see Section~\ref{sec:bibtex} for information on preparing Bib\TeX{} files.

\subsection{Appendices}

Use \verb|\appendix| before any appendix section to switch the section numbering over to letters. See Appendix~\ref{sec:appendix} for an example.

\section{Bib\TeX{} Files}
\label{sec:bibtex}

Unicode cannot be used in Bib\TeX{} entries, and some ways of typing special characters can disrupt Bib\TeX's alphabetization. The recommended way of typing special characters is shown in Table~\ref{tab:accents}.

Please ensure that Bib\TeX{} records contain DOIs or URLs when possible, and for all the ACL materials that you reference.
Use the \verb|doi| field for DOIs and the \verb|url| field for URLs.
If a Bib\TeX{} entry has a URL or DOI field, the paper title in the references section will appear as a hyperlink to the paper, using the hyperref \LaTeX{} package.

\section*{Limitations}
ACL 2023 requires all submissions to have a section titled ``Limitations'', for discussing the limitations of the paper as a complement to the discussion of strengths in the main text. This section should occur after the conclusion, but before the references. It will not count towards the page limit.
The discussion of limitations is mandatory. Papers without a limitation section will be desk-rejected without review.

While we are open to different types of limitations, just mentioning that a set of results have been shown for English only probably does not reflect what we expect. 
Mentioning that the method works mostly for languages with limited morphology, like English, is a much better alternative.
In addition, limitations such as low scalability to long text, the requirement of large GPU resources, or other things that inspire crucial further investigation are welcome.

\section*{Ethics Statement}
Scientific work published at ACL 2023 must comply with the ACL Ethics Policy.\footnote{\url{https://www.aclweb.org/portal/content/acl-code-ethics}} We encourage all authors to include an explicit ethics statement on the broader impact of the work, or other ethical considerations after the conclusion but before the references. The ethics statement will not count toward the page limit (8 pages for long, 4 pages for short papers).

\section*{Acknowledgements}
This document has been adapted by Jordan Boyd-Graber, Naoaki Okazaki, Anna Rogers from the style files used for earlier ACL, EMNLP and NAACL proceedings, including those for
EACL 2023 by Isabelle Augenstein and Andreas Vlachos,
EMNLP 2022 by Yue Zhang, Ryan Cotterell and Lea Frermann,
ACL 2020 by Steven Bethard, Ryan Cotterell and Rui Yan,
ACL 2019 by Douwe Kiela and Ivan Vuli\'{c},
NAACL 2019 by Stephanie Lukin and Alla Roskovskaya, 
ACL 2018 by Shay Cohen, Kevin Gimpel, and Wei Lu, 
NAACL 2018 by Margaret Mitchell and Stephanie Lukin,
Bib\TeX{} suggestions for (NA)ACL 2017/2018 from Jason Eisner,
ACL 2017 by Dan Gildea and Min-Yen Kan, NAACL 2017 by Margaret Mitchell, 
ACL 2012 by Maggie Li and Michael White, 
ACL 2010 by Jing-Shin Chang and Philipp Koehn, 
ACL 2008 by Johanna D. Moore, Simone Teufel, James Allan, and Sadaoki Furui, 
ACL 2005 by Hwee Tou Ng and Kemal Oflazer, 
ACL 2002 by Eugene Charniak and Dekang Lin, 
and earlier ACL and EACL formats written by several people, including
John Chen, Henry S. Thompson and Donald Walker.
Additional elements were taken from the formatting instructions of the \emph{International Joint Conference on Artificial Intelligence} and the \emph{Conference on Computer Vision and Pattern Recognition}.

% Entries for the entire Anthology, followed by custom entries
\bibliography{anthology,custom}
\bibliographystyle{acl_natbib}

\appendix

\section{Example Appendix}
\label{sec:appendix}

This is a section in the appendix.

\end{document}
